\documentclass[11pt]{article}

\usepackage{todonotes}
\usepackage{parskip}

\usepackage[a4paper, top=2.5cm, bottom=2.5cm, left=2.5cm, right=2.5cm]{geometry}
\usepackage{graphicx} % Required for inserting images   
\bibliographystyle{apalike} % use IEEEtran for the numbered IEEE style of citations (remove natbib)
\usepackage{booktabs} 
\usepackage{authblk}
\usepackage{amsmath}
\usepackage{multirow}
\usepackage{ragged2e}
\usepackage{textcomp}
\usepackage{float}
\usepackage{wrapfig}
\usepackage{breqn}
\usepackage{pdfpages}
\usepackage{sectsty}
\usepackage[linesnumbered,ruled]{algorithm2e}
\usepackage{tabto}
\usepackage{afterpage}
\usepackage{hyperref}
\usepackage{natbib} % remove if using IEEE style
\usepackage{colortbl}
\usepackage{mathtools}
\usepackage{titlesec}

% Use setspace for line spacing control
\usepackage{setspace}

% Use standard LaTeX fonts since we're using pdflatex
% Carlito font will be loaded later

\usepackage{caption} 
% Remove duplicate setspace package
\usepackage[sfdefault]{carlito}
\usepackage[breakable, theorems, skins]{tcolorbox}
\tcbset{enhanced}

% ADD THE FOLLOWING COUPLE LINES INTO YOUR PREAMBLE
\let\OLDthebibliography\thebibliography
\renewcommand\thebibliography[1]{
  \OLDthebibliography{#1}
  \setlength{\parskip}{0pt}
  \setlength{\itemsep}{0pt}
}

% filepath: main.tex

% filepath: main.tex

% Section: 11pt, bold, no indent
\titleformat{\section}
{\bfseries\normalsize}           % Bold, 11pt (normalsize in 11pt doc)
{\thesection}{0.5em}             % Label + 0.5em space
{}                               % No extra code before title

% Subsection: 11pt, bold+italic, small indent
\titleformat{\subsection}
{\bfseries\itshape\normalsize}   % Bold + italic, 11pt
{\thesubsection}{1em}            % Label + 1em space
{\hspace{1em}}                   % Indent title by 1em

% Subsubsection: 11pt, italic, larger indent
\titleformat{\subsubsection}
{\itshape\normalsize}            % Italic, 11pt
{\thesubsubsection}{1em}         % Label + 1em space
{\hspace{2em}}                   % Indent title by 2em

% Disable hyperlinks in bibliography
\usepackage{etoolbox}
\makeatletter
\patchcmd{\@bibitem}{\hyper@anchorstart}{\@gobble}{}{}
\patchcmd{\@lbibitem}{\hyper@anchorstart}{\@gobble}{}{}
\makeatother
\renewcommand{\url}[1]{#1}
\usepackage{mathtools}
\usepackage{titlesec}

\DeclareRobustCommand{\mybox}[2][gray!20]{%
\begin{tcolorbox}[   %% Adjust the following parameters at will.
        breakable,
        left=0pt,
        right=0pt,
        top=0pt,
        bottom=0pt,
        colback=#1,
        colframe=#1,
        width=\dimexpr\textwidth\relax, 
        enlarge left by=0mm,
        boxsep=5pt,
        arc=0pt,outer arc=0pt,
        ]
        #2
\end{tcolorbox}
}

\begin{document}

\singlespacing % Reset line spacing to 1 from here on


\mybox{\center{\large\textbf{APPLICATION POSTDOCTORAL FELLOWSHIP (junior/senior)\\
      PROJECT OUTLINE}}}
% The titles below outline mandatory aspects to be included in the project description. You may freely add extra titles and subtitles as required, provided the document does not exceed 10 A4 pages. Please do not alter text layout (Calibri 11 or, when using LaTeX or another word processor, Carlito 11, line spacing 1, all page margins 2.5 cm); this layout applies to all parts of this document (e.g., tables, captions, figures and the reference list). Do not link to external documents or webpages. Remove all explanatory paragraphs (now in comments) from the final document.
\vspace{12pt}

\raggedright
\justifying
\section*{Changes to previous project proposal} %(remove this title if this is the first submission)
% If this postdoctoral project proposal has been submitted to FWO before, highlight any major changes, e.g., how you considered the panel’s feedback to your first application.

\section*{Rationale and positioning with respect to the state of the art}
% Provide a clear problem statement by elaborating on the scientific motivation for the project proposal, based on identified knowledge gaps and issues that the proposed project aims to address. Concisely describe the current state of international research, referencing relevant scientific literature. Position your project in relation to ongoing national and international research efforts. 

%Here is a simple example of citing references. This exaple makes use of both \citet{key} → Author (Year) and \citep{key} → (Author, Year). If you use IEEE style, only \cite{key} → [Number] is available and you have to remove the natbib package. 
Machine learning applications in urban planning have become increasingly important \citep{smith2023machine}, while data science methods continue to evolve for environmental research \citep{anderson2022data}. The development of sustainable transportation networks through multi-objective optimization approaches represents a growing area of research, as shown in \citet{garcia2024sustainable}.

\section*{Scientific research objectives}
% Provide a detailed description of the scientific objectives and, if applicable, the research hypotheses. Elaborate on the distinctive challenges, originality, and inventiveness of the research, with particular emphasis on the innovative aspects of the anticipated results. Thoroughly discuss the expected outcomes (or partial outcomes) that you aim to achieve, including specific knowledge advancements and potential academic breakthroughs.


\section*{Research methodology and work plan}
% Outline the different research steps you envisage and specify the research approach you will use in each step. Motivate why each choice is the best possible way to achieve your objectives. Describe the work plan structure and cohesion. Outline progress monitoring and milestones. Highlight the inventiveness of your research methodology. Address potential scientific risks and contingency plans. Include a table or graphic to represent your planned activities (timing, milestones, critical path) over the 3-year grant period.


% Include all bibliography entries
\bibliography{FWO}
% List relevant bibliographic references preserving text layout (Calibri or Carlito 11, line spacing 1, page margins). Format your references according to the accepted standards in your field and consider including only those that are essential for the assessors’ evaluation of your proposal. 

\end{document}
